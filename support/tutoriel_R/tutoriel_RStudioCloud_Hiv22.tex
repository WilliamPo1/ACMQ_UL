\documentclass[10.5pt,a4paper]{article} 
\usepackage[utf8]{inputenc}  

\usepackage[T1]{fontenc} 
\usepackage[light]{merriweather} 
%\usepackage{ulem}
\usepackage{setspace} 
\usepackage[hang]{footmisc} 
\renewcommand{\footnotesize}{\scriptsize} 
\usepackage[hyphens]{url}\urlstyle{same} 
\usepackage[hidelinks]{hyperref} 
% e-mail
\usepackage{etoolbox}
\makeatletter
\newcommand\myemail[3]{%                %\newcommand\tpj@compose@mailto[3]{%
\edef\@tempa{mailto:#1?subject=#2 }%
\edef\@tempb{\expandafter\html@spaces\@tempa\@empty}%
\href{\@tempb}{#3}}
\catcode\%=11
\def\html@spaces#1 #2{#1%20\ifx#2\@empty\else\expandafter\html@spaces\fi#2}
\catcode\%=14
\makeatother
% Colour
\usepackage{xcolor}
\definecolor{rstudio}{HTML}{4b84b7}
\definecolor{cloud}{HTML}{e4eef8}
\hypersetup{breaklinks=true} 
\Urlmuskip=0mu plus 1mu\relax 
\usepackage[margin=2.5cm]{geometry} 
\usepackage{fancyhdr} % Required for modifying headers and footers
\fancyhead[L]{} % Top left header
\fancyhead[R]{} % Top right header
\renewcommand{\headrulewidth}{1.4pt} % Rule under the header
\fancyfoot[C]{\textbf{\thepage}} % Bottom center footer
\renewcommand{\footrulewidth}{1.4pt} % Rule under the footer
\pagestyle{fancy} % Use the custom headers and footers throughout the document


\usepackage{lipsum,afterpage}
\usepackage{dirtytalk} 
\usepackage{longtable}
\usepackage{adjustbox}
\usepackage{apacite}
\usepackage{natbib}
\usepackage{tkz-euclide}
\usetikzlibrary{calc}
\usepackage{pgfplots}
\pgfplotsset{compat=1.11}
\usepackage {parskip}
\usepackage{epigraph}
\usepackage{graphicx}
\graphicspath{ {images/} }
\pagenumbering{arabic} 
\usepackage{ntheorem}
\newtheorem{hyp}{Hypothesis}
\newtheorem{subhyp}{Hypothesis}[hyp]
\renewcommand\thesubhyp{\thehyp.\alph{subhyp}}
\usepackage{caption}
\usepackage{subcaption}
\usepackage[bf,sf]{titlesec}

\renewcommand{\refname}{Bibliography}
%\setcounter{secnumdepth}{0}
\usepackage[autostyle]{csquotes}
\usepackage{enumitem} % to remove vspace for itemize with [noitemsep]

\usepackage{xparse}
\ExplSyntaxOn

\makeatletter
\NewDocumentCommand{\multicitep}{m}
 {
  \NAT@open
  \mjb_multicitep:n { #1 }
  \NAT@close
 }
\makeatother
\seq_new:N \l_mjb_multicite_in_seq
\seq_new:N \l_mjb_multicite_out_seq
\seq_new:N \l_mjb_cite_seq

\cs_new_protected:Npn \mjb_multicitep:n #1
 {
  \seq_set_split:Nnn \l_mjb_multicite_in_seq { ; } { #1 }
  \seq_clear:N \l_mjb_multicite_out_seq
  \seq_map_inline:Nn \l_mjb_multicite_in_seq
   {
    \mjb_cite_process:n { ##1 }
   }
  \seq_use:Nn \l_mjb_multicite_out_seq { ;~ }
 }

\cs_new_protected:Npn \mjb_cite_process:n #1
 {
  \seq_set_split:Nnn \l_mjb_cite_seq { , } { #1 }
  \int_compare:nTF { \seq_count:N \l_mjb_cite_seq == 1 }
   {
    \seq_put_right:Nn \l_mjb_multicite_out_seq
     { \citeauthor{#1},~\citeyear{#1} }
   }
   {
    \seq_put_right:Nx \l_mjb_multicite_out_seq
     {
      \exp_not:N \citeauthor{\seq_item:Nn \l_mjb_cite_seq { 1 }},~
      \exp_not:N \citeyear{\seq_item:Nn \l_mjb_cite_seq { 1 }},~
      \seq_item:Nn \l_mjb_cite_seq { 2 }
     }
   }
 }
\ExplSyntaxOff

\title{Utiliser \textbf{\emph{RStudio Cloud}} pour les nuls}
\author{William Poirier, Université Laval}
\date{POL-2000 -- Août 2021}

\begin{document} 

% ----------------------------------------------------------------
\begin{titlepage}
\newgeometry{left=7.5cm} %defines the geometry for the titlepage
\pagecolor{rstudio}
\noindent
\includegraphics[width=13.6cm]{cloud.jpg}\\[-1em]
\color{cloud}
\makebox[0pt][l]{\rule{1.4\textwidth}{2pt}}
\par
\noindent
\color{white}
\textbf{Université Laval}
\vfill
\noindent
{\huge Utiliser \textbf{\emph{RStudio Cloud}} pour les nuls}
\vskip\baselineskip
\noindent
POL-2000 -- Août 2021
\end{titlepage}
\restoregeometry % restores the geometry
\nopagecolor% Use this to restore the color pages to white
% ----------------------------------------------------------------

\tableofcontents

\pagebreak

\section{C'est quoi R?}
  \subsection{Recomendations}
Something\footnote{Tutoriel basé sur le travail de Vincent Arel-Bundock, Yannick Dufresne, et Florence Vallée-Dubois}
\section{R vs RStudio vs RStudio Cloud}

\section{RStudio Cloud - La base}
  \subsection{Connexion}
  \subsection{Interface}
  \subsection{Projets}

\section{Commandes de base}
  \subsection{Workflow}
    \subsubsection{Arborescence}
    \subsubsection{Commentaires}
    \subsubsection{Raccourcis clavier}
  \subsection{Mathématiques}
    \subsubsection{Calculatrice}
    \subsubsection{Calculatrice+ (table,mean,sd,var etc.}
    \subsubsection{Opérateurs de comparaison}


\section{Intro à la programmation en R}
  \subsection{Ça va bien aller}
  \subsection{L'assignation d'objets}
    \subsubsection{Un peu de vocabulaire}
  \subsection{Les types d'objets}
    \subsubsection{Classes d'éléments}
    \subsubsection{Constante}
    \subsubsection{Vecteurs}
    \subsubsection{Dataframe}
  \subsection{Normes d'assignation - C'est pour votre bien}

  
\section{Importation de données}
  \subsection{Logique de projet RStudio Cloud}
  \subsection{Type de fichier}
  \subsection{Domestication}

  
\section{Analyse - Description}
  \subsection{Univariée}
  \subsection{Bivariée}
  \subsection{Visualisation}


\section{Analyse - Régression}
  \subsection{Linéaire}
  \subsection{Linéaire multiple}

  
\section{Pour aller plus loins}
 
%\section{Bibliographie}
%\begingroup
%\renewcommand{\section}[2]{}
%\bibliographystyle{apacite}
%\bibliography{mybibfile.bib}
%\vspace{10cm}
%\endgroup

\end{document}